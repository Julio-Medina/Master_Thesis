\documentclass[a4paper]{article}
\usepackage[spanish,es-tabla]{babel}	% trabajar en español
\spanishsignitems	
%\usepackage{simplemargins}

%\usepackage[square]{natbib}
\usepackage{amsmath}
\usepackage{amsfonts}
\usepackage{amssymb}
\usepackage{bbold}
\usepackage{graphicx}
\usepackage{blindtext}
\usepackage{hyperref}
\usepackage{mathtools}
\usepackage{dirtytalk}
\usepackage{booktabs}
\usepackage{algorithm}
%\usepackage{algorithmic}
\usepackage{algpseudocode}
\usepackage{fancyvrb}

\begin{document}
\pagenumbering{arabic}

\Large
 \begin{center}
\textbf{Computación cuántica y las desigualdades de Bell}\\
Protocolo de Tesis

\hspace{10pt}

% Author names and affiliations
\large
Lic. Julio A. Medina$^1$ \\

\hspace{10pt}
\small  
$^1$ Universidad de San Carlos, Escuela de Ciencias Físicas y Matemáticas\\
Maestría en Física\\
\href{mailto:julioantonio.medina@gmail.com}{julioantonio.medina@gmail.com}\\

\end{center}

\hspace{10pt}


\normalsize

%\begin{abstract}

%\end{abstract}

\section{Introducción}
Varias aplicaciones de la mecánica cuántica han tenido gran relevancia en el desarrollo científico y tecnológico de la sociedad moderna. La computación cuántica es uno de esas aplicaciones con un gran potencial en diversos tipos de problemas que van desde la misma simulación de sistemas intrínsecamente cuánticos cómo simulaciones financieras y la resolución de problemas de optimización complejos. \\

Los sistemas cuánticos proveen un paradigma de computo totalmente disruptivo en comparación a la computación clásica. Esto es posible gracias a dos propiedades fundamentales de los sistemas cuánticos, i.e. la superposición de estados y el \textbf{entrelazamiento de estados}. La unidad fundamental de la computación cuántica es el \textbf{qubit} que gracias a la superposición de estados y al entrelazamiento cuántico puede configurarse a sistemas de qubits entrelazados para que realicen operaciones en varios estados simultáneamente o en paralelo dando una ventaja o supremacía sobre la computación clásica. Un ejemplo bien conocido es el algoritmo de factorización de Shor.
\section{Justificación}
La motivación inicial para el desarrollo de dispositivos capaces de simular sistemas cuánticos viene de un artículo seminal de R. Feynman \cite{Feynman}, en donde expone cómo inevitablemente se tienen que recurrir a sistemas de naturaleza cuántica para poder hacer simulaciones de sistemas cuánticos ya que conforme aumenta el número de partículas en dichos sistemas la complejidad algorítmica crece exponencialmente, algo que hace el estudio pragmático de estos sistemas sea un gran reto tecnológico para sistemas clásicos, sobretodo cuando se trata de problemas de varios cuerpos.
\section{Objetivos}
\section{Marco Teórico}
\section{Metodología}

\begin{thebibliography}{99}
%% La bibliografía se ordena en orden alfabético respecto al apellido del 
%% autor o autor principal
%% cada entrada tiene su formatado dependiendo si es libro, artículo,
%% tesis, contenido en la web, etc
\bibitem{Arfken} George Arfken. \textit{Mathematical Methods for Physicists}.

\bibitem{Bell} J.S. Bell. \textit{On the Einstein Podolski Rosen Paradox}. \url{https://cds.cern.ch/record/111654/files/vol1p195-200_001.pdf}

\bibitem{Clauser} John F. Clauser, Michael A. Horne, Abner Shimony, Richard Holt. \textit{PROPOSED EXPERIMENT TO TEST LOCAL HIDDEN-VARIABLE THEORIES.}. Physical Review Letters,. 23(15):880-4, \url{https://journals.aps.org/prl/abstract/10.1103/PhysRevLett.23.880}.

\bibitem{Einstein} Einstein A., B. Podolsky, N. Rosen, \textit{Can Quantum-Mechanical Description of Physical Reality be Considered Complete?}. Physical Review. \url{doi:10.1103/PhysRev.47.777}

\bibitem{Medina} Julio Medina. \textit{Reporte de Seminario 1. Computación Cuántica}. \url{https://github.com/Julio-Medina/Seminario/blob/main/Reporte_final/reporte_final.pdf}

\bibitem{Nielsen} Michael A. Nielsen, Isaac L. Chuang. \textit{Quantum Computation adn Quantum Information}. Cambridge University Press 2010. 10th. Anniversary Edition.

\bibitem{Feynman} Richard P. Feynman. \textit{Simulating Physics with Computers.} \url{https://doi.org/10.1007/BF02650179}.

\bibitem{Qiskit} \textit{Qiskit Textbook}. \url{https://qiskit.org/textbook-beta}

\bibitem{Mermin} N. David Mermin \textit{Quantum Computer Science: An Introduction}. Cambridge University Press, 2007.

\bibitem{Sakurai} J.J. Sakurai \textit{Modern Quantum Mechanics}. The Benjamin/Cummings Publishing Company, 1985.

\bibitem{Dotsenko} Viktor Dotsenko. \textit{An Introduction to the Theory of Spin Glasses and Neural Networks}. World Scientific 1994.

\bibitem{Bahri} Yasaman Bahri, Jonathan Kadmon, Jeffrey Pennington, Sam S. Schoenholz, Jascha Sohl-Dickstein, Surya Ganguli. \textit{Statistical Mechanics of Deep Learning}. \url{https://www.annualreviews.org/doi/pdf/10.1146/annurev-conmatphys-031119-050745}

\bibitem{openQASM} OpenQASM. \url{https://github.com/openqasm/openqasm}.
 

\end{thebibliography}
\end{document}

