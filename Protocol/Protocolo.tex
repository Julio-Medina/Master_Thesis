\documentclass[a4paper]{article}
\usepackage[spanish,es-tabla]{babel}	% trabajar en español
\spanishsignitems	
%\usepackage{simplemargins}

%\usepackage[square]{natbib}
\usepackage{amsmath}
\usepackage{amsfonts}
\usepackage{amssymb}
\usepackage{bbold}
\usepackage{graphicx}
\usepackage{blindtext}
\usepackage{hyperref}
\usepackage{mathtools}
\usepackage{dirtytalk}
\usepackage{booktabs}
\usepackage{algorithm}
%\usepackage{algorithmic}
\usepackage{algpseudocode}
\usepackage{fancyvrb}

\begin{document}
\pagenumbering{arabic}

\Large
 \begin{center}
\textbf{Computación cuántica y las desigualdades de Bell}\\
Protocolo de Tesis

\hspace{10pt}

% Author names and affiliations
\large
Lic. Julio A. Medina$^1$ \\

\hspace{10pt}
\small  
$^1$ Universidad de San Carlos, Escuela de Ciencias Físicas y Matemáticas\\
Maestría en Física\\
\href{mailto:julioantonio.medina@gmail.com}{julioantonio.medina@gmail.com}\\

\end{center}

\hspace{10pt}


\normalsize

%\begin{abstract}

%\end{abstract}

\section{Introducción}
Varias aplicaciones de la mecánica cuántica han tenido gran relevancia en el desarrollo científico y tecnológico de la sociedad moderna. La computación cuántica es una de esas aplicaciones con un gran potencial en diversos tipos de problemas que van desde la misma simulación de sistemas intrínsecamente cuánticos hasta simulaciones financieras y la resolución de problemas de optimización complejos. \\

Los sistemas cuánticos proveen un paradigma de computo totalmente disruptivo en comparación a la computación clásica. Esto es posible gracias a dos propiedades fundamentales de los sistemas cuánticos, i.e. la superposición de estados y el \textbf{entrelazamiento de estados}. La unidad fundamental de la computación cuántica es el \textbf{qubit} que gracias a la superposición de estados y al entrelazamiento cuántico puede configurarse a sistemas de qubits entrelazados para que realicen operaciones en varios estados simultáneamente o en paralelo dando una ventaja o supremacía sobre la computación clásica. Un ejemplo bien conocido es el algoritmo de factorización de Shor.
\section{Justificación}
La motivación inicial para el desarrollo de dispositivos capaces de simular sistemas cuánticos viene de un artículo seminal de R. Feynman \cite{Feynman}, en donde expone cómo inevitablemente se tienen que recurrir a sistemas de naturaleza cuántica para poder hacer simulaciones de sistemas cuánticos ya que conforme aumenta el número de partículas en dichos sistemas la complejidad algorítmica crece exponencialmente, algo que hace el estudio pragmático de estos sistemas sea un gran reto tecnológico para sistemas clásicos, sobretodo cuando se trata de problemas de varios cuerpos.\\

En esta investigación se quiere tener un entendimiento fundamental de los conceptos teóricos de la computación cuántica, además de complementarlo con aplicaciones de software específicamente utilizando las herramientas programáticas creadas por \textit{IBM, Qiskit}. Con esto se quiere realizar una simulación en una computadora cuántica real para revisar un resultado famoso que sirve para confirmar las desigualdades de Bell, i.e. las desigualdades CHSH.\\

También se investigaran temas relacionados cómo el \textit{Quantum Machine Learning}, esto para poder entender cómo los ordenadores cuánticos pueden potenciar los ya exitosos sistema de Inteligencia Artificial. Tener un entendimiento fundamental de estas tecnologías es necesario para poder a empezar a buscar aplicaciones que puedan impulsar el desarrollo tecnológico.
\section{Objetivos}
\begin{enumerate}
\item Comprender los principios básicos de la computación cuántica a través de una revisión detallada de mecánica cuántica avanzada.
\item Derivar las desigualdades de Bell para entender su significancia.
\item Familiarizarse con las herramientas teóricas para el estudio de computación cuántica.
\item Entender las implementaciones básicas de los algoritmos cuánticos existentes, por nombrar algunos: el Algoritmo de Shor, el Algoritmo de Deutch-Jozza.
\item Traducir los algoritmos cuánticos de su formulación matemática a una implementación en \texttt{Python} utilizando \textit{Qiskit}.
\item Analizar la relación entre las desigualdades de Bell y las desigualdades CHSH.
\item Utilizar un ordenar cuántico real para comprobar las desigualdades CHSH y de esa manera comprobar experimentalmente el teorema de Bell.

\end{enumerate}
\section{Marco Teórico}
\subsection{Notación de Dirac}
Se utilizan para describir estados cuánticos. Son vectores en el espacio de Hilbert, pueden ser vectores con dimensiones finitas o describir cantidades continuas y ser vectores de dimensión infinita. El físico inglés Paul Dirac popularizó la llamada notación de Dirac para hacer los cálculos en mecánica cuántica mas compactos.
Se define al ket como
\begin{equation}
|a\rangle=
	\begin{pmatrix}
		a_1\\
		a_2
	\end{pmatrix}
\end{equation}
donde $a_1, a_2 \in \mathbb{C} $. Y el bra se define como el vector fila, o vector en el espacio dual del ket definido como 
\begin{equation}
\langle a|=
	\begin{pmatrix}
		a_1^*&a_2^*
	\end{pmatrix}
\end{equation}
donde $a_1^*, a_2^* \in \mathbb{C} $ son los complejos conjugados de $a_1, a_2$.
Los bra y los ket están relacionados de la siguiente manera
\begin{equation}
\langle a|=|a\rangle^\dagger
\end{equation}
que equivale a 
\begin{equation}
\begin{pmatrix}
		a_1\\
		a_2
	\end{pmatrix}^\dagger=
	\begin{pmatrix}
		a_1^*&a_2^*
	\end{pmatrix}
\end{equation}
donde $a_1=x+iy$ y $a_1^*=x-iy$ es el complejo conjugado, el operador $\dagger$ da el transpuesto conjugado de un vector o matriz.\\
%\subsubsection{BraKet}
Se define al braket como el producto interno o producto escalar en el espacio de Hilbert.
\begin{equation}
\langle b | a\rangle=a_1b_1^*+a_2b_2^*=\langle a | b\rangle^*\in \mathbb{C}
\end{equation}.
El ketbra o producto abierto se define como 
\begin{equation}
|a\rangle\langle b|=
	\begin{pmatrix}
		a_1 b_1^*& a_1 b_2^*\\
		a_2 b_1^*& a_2 b_2^*\\
	\end{pmatrix}
\end{equation}
\subsection{Representación de Qubit}
La representación de un qubit viene dada naturalmente por los spinores utilizados para describir el experimento de Stern-Gerlach en el que se miden el spin de átomos de plata. Estos spinores se definen como 
\begin{equation}\label{eq::qubit0}
|0\rangle\equiv
\begin{pmatrix}
		1\\
		0
	\end{pmatrix}
\end{equation}
como el análogo cuántico del estado $0$ para un bit clásico, y para el estado clásico $1$ se usa
\begin{equation}\label{eq::qubit1}
|1\rangle\equiv
\begin{pmatrix}
		0\\
		1
	\end{pmatrix}
\end{equation}
estos estados cuánticos son evidentemente ortonormales $\langle 0 | 1\rangle=0$ y $\langle 0 | 0\rangle=\langle 1 | 1\rangle=1$. Se nota que estos dos estados forman una base. Todos los estados cuánticos están normalizados
\begin{equation}
\langle \psi | \psi\rangle=1
\end{equation}
por ejemplo $|\psi\rangle=\frac{1}{\sqrt{2}}(|0\rangle+|1\rangle)$.
\subsection{Mediciones}
Por lo general se utilizan bases ortonormales para describir y medir estados cuánticos, aquí se sigue esa convención. Durante una medición en la base $\{|0\rangle, |1\rangle\}$ el estado colapsara ya sea a $|0\rangle$ o $|1\rangle$, estos son los eigen-estados de $\sigma_z$ la matriz Pauli-z(ver \ref{sec::PauliMatrix}) a esto se le conoce como medición en z.\\
Hay infinitas bases distintas en las que se pueden realizar las mediciones pero otras bases comunes son $\{|+\rangle, |-\rangle \}$ definidas como:
\begin{equation}
|+\rangle\equiv\frac{1}{\sqrt{2}}(|0\rangle+|1\rangle)
\end{equation}
\begin{equation}
|-\rangle\equiv\frac{1}{\sqrt{2}}(|0\rangle-|1\rangle)
\end{equation}
que son los eigen-vectores de la matriz de Pauli x $\sigma_x$, y como se ha de esperar otra base comúnmente usada son los eigen-vectores de la matriz Pauli $y$, $\sigma_y$:
\begin{equation}
|+i\rangle\equiv\frac{1}{\sqrt{2}}(|0\rangle+i|1\rangle)
\end{equation}
\begin{equation}
|-i\rangle\equiv\frac{1}{\sqrt{2}}(|0\rangle-i|1\rangle)
\end{equation}
\subsection{Regla de Born}
La probabilidad que el estado $|\psi\rangle$ colapse durante una medición proyectiva en la base $\{|x\rangle, |x^+ \rangle \}$ al estado $|x\rangle$ está dada por
\begin{equation}
P(x)=|\langle x|\psi\rangle|^2
\end{equation}
y del hecho que se está trabajando con estados normalizados se tiene, como es de esperarse que
\begin{equation}
\sum_i P(x_i)=1
\end{equation}
Para ilustrar esto se presentan algunos ejemplos:\\
a. El estado $|\psi\rangle=\frac{1}{\sqrt{3}}(|0\rangle +\sqrt{2}|1\rangle)$ es medido en la base $\{|0\rangle,   |1\rangle \}$ calcule las probabilidades correspondientes:
\begin{equation*}
P(0)=\bigg |\langle 0|\frac{1}{\sqrt{3}}(|0\rangle +\sqrt{2}|1\rangle)\bigg |^2=\bigg|\frac{1}{\sqrt{3}}\langle 0|0\rangle+\sqrt{\frac{2}{3}}\langle 0|1\rangle \bigg |^2
\end{equation*}
\begin{equation*}
P(0)=\bigg|\frac{1}{\sqrt{3}} \bigg|^2=\frac{1}{3}
\end{equation*}
\begin{equation*}
P(1)=\bigg|\sqrt{\frac{2}{3}} \bigg|^2=\frac{2}{3}
\end{equation*}
\begin{equation*}
P(0)+P(1)=1
\end{equation*}
b. El estado $|\psi\rangle=\frac{1}{\sqrt{2}}(|0\rangle-|1\rangle)$ es medido en la base $\{|+\rangle, |-\rangle\}$, encontrar las probabilidades correspondientes:
\begin{equation*}
P(+)=\bigg| \langle +|\psi\rangle\bigg|^2=\bigg|\frac{1}{\sqrt{2}}(\langle 0| +\langle 1|)\cdot \frac{1}{\sqrt{2}}(|0\rangle -|1\rangle|)\bigg|^2
\end{equation*}
\begin{equation*}
P(+)=\frac{1}{4}\bigg|\langle 0|0\rangle-\langle 0|1\rangle+\langle 1|0\rangle-\langle 1|1\rangle\bigg|^2=0
\end{equation*}
\begin{equation*}
P(-)=1
\end{equation*}
Como es de esperarse ya que $\bigg| \langle -|\psi\rangle\bigg|^2=\bigg| \langle -|-\rangle\bigg|^2=1$ y $\bigg| \langle +|-\rangle\bigg|^2=0$ por la propiedad de ortogonalidad de estas bases.

\subsection{Transformada Cuántica de Fourier}
\subsubsection{Introducción}
Como se mencionó antes la transformada de Fourier tienen mucha aplicaciones en toda la física y también tiene aplicaciones importantes en la computación clásica en áreas como el procesamiento de señales, algoritmos de compresión de datos y teoría de complejidad de algoritmos. La transformada de Fourier Cuántica es la implementación cuántica de la transformada discreta de Fourier sobre las amplitudes de una función de onda\footnote{Una función de onda que evoluciona de acuerdo a la ecuación de Schr\"{o}dinger}, es una parte fundamental de varios algoritmos cuánticos  como el algoritmo de factorización de Shor y la estimación de fase cuántica.\\
La transformada de Fourier cuántica actúa sobre un vector $(x_0, \hdots, x_{N-1})$ y lo mapea o transforma o otro vector
$(y_0, \hdots, y_{N-1})$ por medio de la siguiente fórmula
\begin{equation}
y_{k}=\frac{1}{\sqrt{N}}\sum_{j=0}^{N-1}x_j \omega_{N}^{jk}
\end{equation}
donde 
\begin{equation}\label{eq::omega_definition}
\omega_{N}^{jk}=e^{2\pi i \frac{jk}{N}}.
\end{equation}

De manera análoga la transformada cuántica de Fourier actúa sobre un estado cuántico\footnote{Un vector en un espacio de Hilbert, finito o infinito} 
\begin{equation}
|X\rangle=\sum_{j=0}^{N-1} x_{j}|j\rangle
\end{equation}
y lo mapea a un estado 
\begin{equation}\label{eq::QFT_general}
|Y\rangle=\sum_{k=0}^{N-1} y_{k}|k\rangle
\end{equation}
conforme a la expresión
\begin{equation}
y_k=\frac{1}{\sqrt{N}}\sum_{j=0}^{N-1}x_j\omega_{N}^{jk}
\end{equation}
donde $\omega_{N}^{jk}$ está definido en \ref{eq::omega_definition}, es importante notar que sólo las amplitudes del estado son afectadas por está transformación, esto puede expresarse también como el mapeo
\begin{equation}
|j\rangle \rightarrow \frac{1}{\sqrt{N}}\sum^{N-1}_{k=0}\omega_N^{jk}|k\rangle
\end{equation}
o por la matriz unitaria
\begin{equation}
U_{\text{QFT}}=\frac{1}{\sqrt{N}}\sum_{j=0}^{N-1}\sum_{k=0}^{N-1}\omega_{N}^{jk}|k\rangle\langle j|
\end{equation}
\subsubsection{Intuición detrás de la TCF(QFT)}
La intuición detrás de la Transformada Cuántica de Fourier(QFT-\textit{Quantum Fourier Transform}) viene del hecho que el efecto que tiene al transformar a un estado cuántico basicamente se está cambiando de la base computacional Z  a la base de Fourier. La compuerta de Hadamard(\cite{Medina}, \cite{Qiskit}, \cite{Nielsen}) es la transformada de Fourier cuántica para un solo qubit, transforma de base computacional Z $\{|0\rangle, |1\rangle\}$ a la base X $\{|+\rangle, |-\rangle \}$. De la misma manera todos los estados de partículas múltiples es decir estados de varios qubit en la base computacional tiene una representación en la base de Fourier. La TCF simplemente es la función que transforma entre estas bases.
\begin{equation}
\text{QFT}|x\rangle = |\tilde{x}\rangle
\end{equation}
donde la base de Fourier se denota como $|\tilde{x}\rangle$.
\subsubsection{Ejemplo 1: TCF de 1-qubit}
Se empieza por considerar como el operador QFT definido previamente, actúa sobre  un estado de un solo qubit, i.e. $|\psi\rangle=\alpha|0\rangle+\beta |1\rangle$. Para este caso $x_0=\alpha, x_1=\beta$, y $N=2$, con esto se obtiene
que el primer coeficiente de Fourier en la transformación es 
\begin{equation}
y_0=\frac{1}{\sqrt{2}}\Bigg( \alpha \exp\bigg( 2\pi i \frac{0\times 0}{2}{•} \bigg) + \beta \exp\bigg( 2\pi i \frac{1\times 0}{2}{•} \bigg) \Bigg)=\frac{1}{\sqrt{2}}(\alpha+\beta)
\end{equation}
y para el segundo coeficiente se tiene
\begin{equation}
y_1=\frac{1}{\sqrt{2}}\Bigg( \alpha \exp\bigg( 2\pi i \frac{0\times 1}{2}{•} \bigg) + \beta \exp\bigg( 2\pi i \frac{1\times 1}{2}{•} \bigg) \Bigg)=\frac{1}{\sqrt{2}}(\alpha-\beta)
\end{equation}
con esto se obtiene el resultado total sobre $|\psi\rangle$
\begin{equation}
U_{\text{QFT}}|\psi\rangle=\frac{1}{\sqrt{2}}(\alpha+\beta)|0\rangle + \frac{1}{\sqrt{2}}(\alpha-\beta)|1\rangle
\end{equation}
está operación es equivalente a aplicar el operador de Hadamard
\begin{equation}\label{eq::Hadamard_gate}
H=\frac{1}{\sqrt{2}}
\begin{bmatrix}
1&1\\
1&-1
\end{bmatrix}
\end{equation}
al estado $\vert \psi\rangle$, 
\begin{equation}
\begin{aligned}
H\vert\psi\rangle&=\frac{1}{\sqrt{2}}(\alpha+\beta)|0\rangle + \frac{1}{\sqrt{2}}(\alpha-\beta)|1\rangle\equiv \tilde{\alpha}\vert 0\rangle+ \tilde{\beta}\vert 1\rangle\\
&=\alpha \vert + \rangle +\beta \vert - \rangle
\end{aligned}
\end{equation}
\subsubsection{Transformada Cuántica de Fourier: definición general}
La ejemplo anterior no deja del todo claro el como se aplica una transformada cuántica de Fourier para N más grandes. Aquí se toma a $N=2^n$, es decir $U_{\text{QFT}_N}$ actuando en el estado $|x\rangle=|x_1\hdots x_n\rangle$\footnote{Se hace uso de la notación $|x_1\hdots x_n\rangle=|x_1\rangle\otimes\hdots\otimes |x_n\rangle$ } donde $x_1$ es el bit más significativo. Con esto se aplica la expresión \ref{eq::QFT_general} para obtener
\begin{equation}\label{eq::QFT}
\begin{aligned}
U_{\text{QFT}_N}|x\rangle &=\frac{1}{\sqrt{N}}\sum_{y=0}^{N-1}\omega_N^{xy}|y\rangle\\
&=\frac{1}{\sqrt{N}}\sum_{y=0}^{N-1} e^{2\pi i xy/2^{n}}|y\rangle
\end{aligned}
\end{equation}
reescribiendo $y$ en notación fraccional binaria $y=y_1\hdots y_n$ con
\begin{equation*}
\frac{y}{2^n}=\sum_{k=1}^{n}\frac{y_k}{2^k}
\end{equation*}
donde $y_k \in\{ 0, 1 \}$, sustituyendo en \ref{eq::QFT} se obtiene
\begin{equation}\label{eq::QFT_2}
\begin{aligned}
U_{\text{QFT}_N}|x\rangle &=\frac{1}{\sqrt{N}}\sum_{y=0}^{N-1}e^{2\pi i(\sum_{k=1}^n y_k/2^k)}|y_1 \hdots y_n\rangle\\
&=\frac{1}{\sqrt{N}}\sum_{y=0}^{N-1} \prod_{k=1}^n e^{2\pi i x y_k/2^{k}}|y_1 \hdots y_n\rangle
\end{aligned}
\end{equation}
donde se ha expresado al exponencial de una suma como un producto de exponenciales. Re-arreglando las sumas y productos, y expandiendo $\sum_{y=0}^{N-1}=\sum_{y_1=0}^1 \sum_{y_2=0}^1 \hdots \sum_{y_n=0}^1$, se obtiene
\begin{equation}\label{eq::QFT_3}
\begin{aligned}
U_{\text{QFT}_N}|x\rangle &=\frac{1}{\sqrt{N}} \bigotimes_{k=1}^n \bigg(|0\rangle + e^{2\pi i x/2^k}|1\rangle \bigg)  \\
&=\frac{1}{\sqrt{N}}\bigg(|0\rangle + e^{\frac{2\pi i}{2} x} |1\rangle \bigg)\otimes\bigg(|0\rangle + e^{\frac{2\pi i}{2^2} x} |1\rangle \bigg)\otimes \hdots \otimes \bigg(|0\rangle + e^{\frac{2\pi i}{2^{n}} x} |1\rangle \bigg)
\end{aligned}
\end{equation}
\section{Metodología}
La metodología a usar es un abordaje híbrido en el que se estudian a fondo los temas teóricos y se van aplicando a una implementación básica en \textit{Qiskit}, estás implementaciones se van a ir versionando en un repositorio de \texttt{GitHub} para poder reproducir todos los resultados.\\

Los resultados teóricos se desarrollan desde principios fundamentales y se dejará constancia de las deducciones realizadas, los algoritmos se van a ir probando por medio de la interfaz que ofrece \textit{IBM} para utilizar ordenares cuánticos reales, esto para cuantificar los efectos de los errores en los algoritmos desarrollados. También se investigará cómo el \textit{Quantum Machine Learning} puede utilizarse para potenciar al \textit{Machine Learning} clásico, esto servirá como base para escribir una sección en un artículo sobre un modelo de \textit{Machine Learning} llamado SeismicNet. SeismicNet es un modelo red neuronal profunda para el análisis automático de señales sísmicas.



\begin{thebibliography}{99}
%% La bibliografía se ordena en orden alfabético respecto al apellido del 
%% autor o autor principal
%% cada entrada tiene su formatado dependiendo si es libro, artículo,
%% tesis, contenido en la web, etc
\bibitem{Arfken} George Arfken. \textit{Mathematical Methods for Physicists}.

\bibitem{Bell} J.S. Bell. \textit{On the Einstein Podolski Rosen Paradox}. \url{https://cds.cern.ch/record/111654/files/vol1p195-200_001.pdf}

\bibitem{Clauser} John F. Clauser, Michael A. Horne, Abner Shimony, Richard Holt. \textit{PROPOSED EXPERIMENT TO TEST LOCAL HIDDEN-VARIABLE THEORIES.}. Physical Review Letters,. 23(15):880-4, \url{https://journals.aps.org/prl/abstract/10.1103/PhysRevLett.23.880}.

\bibitem{Einstein} Einstein A., B. Podolsky, N. Rosen, \textit{Can Quantum-Mechanical Description of Physical Reality be Considered Complete?}. Physical Review. \url{doi:10.1103/PhysRev.47.777}

\bibitem{Medina} Julio Medina. \textit{Reporte de Seminario 1. Computación Cuántica}. \url{https://github.com/Julio-Medina/Seminario/blob/main/Reporte_final/reporte_final.pdf}

\bibitem{Nielsen} Michael A. Nielsen, Isaac L. Chuang. \textit{Quantum Computation adn Quantum Information}. Cambridge University Press 2010. 10th. Anniversary Edition.

\bibitem{Feynman} Richard P. Feynman. \textit{Simulating Physics with Computers.} \url{https://doi.org/10.1007/BF02650179}.

\bibitem{Qiskit} \textit{Qiskit Textbook}. \url{https://qiskit.org/textbook-beta}

\bibitem{Mermin} N. David Mermin \textit{Quantum Computer Science: An Introduction}. Cambridge University Press, 2007.

\bibitem{Sakurai} J.J. Sakurai \textit{Modern Quantum Mechanics}. The Benjamin/Cummings Publishing Company, 1985.

\bibitem{Dotsenko} Viktor Dotsenko. \textit{An Introduction to the Theory of Spin Glasses and Neural Networks}. World Scientific 1994.

\bibitem{Bahri} Yasaman Bahri, Jonathan Kadmon, Jeffrey Pennington, Sam S. Schoenholz, Jascha Sohl-Dickstein, Surya Ganguli. \textit{Statistical Mechanics of Deep Learning}. \url{https://www.annualreviews.org/doi/pdf/10.1146/annurev-conmatphys-031119-050745}

\bibitem{openQASM} OpenQASM. \url{https://github.com/openqasm/openqasm}.
 

\end{thebibliography}
\end{document}

